% !TEX TS-program = xelatex
%
% Created by Gerson Sunyé on 2016-12-14.
% Copyright (c) 2016 .
\documentclass[a4paper,11pt]{memoir}
\usepackage[]{xsim}
\usepackage[utf8]{inputenc}
\usepackage{lmodern}
\usepackage{listings,multicol}
\usepackage{hyperref}
\usepackage{graphicx}
\usepackage[svgnames]{xcolor}
\usepackage{polyglossia}
\setdefaultlanguage{english}
\usepackage{pdfpages}
\usepackage{typehtml}
\usepackage{lscape}
\usepackage[french,boxed,lined]{algorithm2e}
\usepackage{amsmath}
\usepackage{siunitx}
\xsimsetup{solution/print=false}
\usepackage{booktabs}
\usepackage{paralist}


\lstset{basicstyle=\footnotesize,
	keywordstyle=\bfseries,
	stringstyle=\ttfamily,
	showstringspaces=\false,
	frame=tb,
	commentstyle=\itshape\color{magenta},
	extendedchars=true,
	showtabs=false,
	tabsize=2,
	tab=$\mapsto$,
	numbers=left, 
	numberstyle=\tiny,
	sensitive,
	texcl,
	breaklines=true,
	literate={é}{e}{1} {à}{a}{1}
	}

\lstnewenvironment{java}{\lstset{language=Java,        
        flexiblecolumns,
		keywordstyle=\bfseries,
        sensitive, 
		texcl
        }}{}

\lstnewenvironment{ocl}{\lstset{language=[decorative]OCL,
	frame=tb,
	tabsize=3,
	morekeywords={implies,result,flatten,body,init,OrderedSet,self,Tuple,TupleType,def,attr,oclIsUndefined,oclIsInvalid,OclState,let,in},
	morecomment=[l]{--},
	basicstyle=\footnotesize,
	keywordstyle=\bfseries,
	ndkeywordstyle=\bfseries,
	commentstyle=\itshape,
	stringstyle=\ttfamily,
	showspaces=false,
	flexiblecolumns,
	literate={->}{$\to$}{2} {--}{-$\,$-}{2} {<=}{$\le$}{2} {>=}{$\ge$}{2} {<>}{$<\,>$}{3},
	sensitive, extendedchars, texcl}}{}


		
\newcommand{\code}[1]{\lstinline{#1}} 
\graphicspath{{./img/}}

\chapterstyle{ell}


% \firstpageheader{Test logiciel}{exercises}{Page \thepage\ of \numpages}
% \firstpageheadrule

\title{{\Huge\bfseries Software Construction and Evolution} \\[5pt] {\Large\bfseries Instructor-led exercises}}
\date{}
\author{Gerson Sunyé \\ \href{mailto:gerson.sunye@univ-nantes.fr}{gerson.sunye@univ-nantes.fr} \\ Université de Nantes
\and Gilles Ardourel \\ \href{matilto:gilles.ardourel@univ-nantes.fr}{gilles.ardourel@univ-nantes.fr} \\ Université de Nantes
}

\newcommand*{\plogo}{\fbox{$\mathcal{PL}$}} % Generic publisher logo
\newcommand*{\rotrt}[1]{\rotatebox{90}{#1}} % Command to rotate right 90 degrees
\newcommand*{\rotlft}[1]{\rotatebox{-90}{#1}} % Command to rotate left 90 degrees
\newcommand*{\titleBC}{\begingroup % Create the command for including the title page in the document
\centering % Center all text

\def\CP{\textit{\Huge Software Construction and Evolution}} % Title

\settowidth{\unitlength}{\CP} % Set the width of the curly brackets to the width of the title
{\color{LightGoldenrod}\resizebox*{\unitlength}{\baselineskip}{\rotrt{$\}$}}} \\[\baselineskip] % Print top curly bracket
\textcolor{Sienna}{\CP} \\[\baselineskip] % Print title
{\color{RosyBrown}\Large Instructor-led exercises} \\ % Tagline or further description
{\color{LightGoldenrod}\resizebox*{\unitlength}{\baselineskip}{\rotlft{$\}$}}} % Print bottom curly bracket

\vfill % Whitespace between the title and the author name

%{\textbf{John Smith}}\\ % Author name

{\Large \textbf{Gerson Sunyé} \\ \href{mailto:gerson.sunye@univ-nantes.fr}{gerson.sunye@univ-nantes.fr} \\ Université de Nantes} \\[10pt]
{\Large \textbf{Gilles Ardourel} \\ \href{matilto:gilles.ardourel@univ-nantes.fr}{gilles.ardourel@univ-nantes.fr} \\ Université de Nantes} \\[10pt]


\vfill % Whitespace between the author name and the publisher logo

\plogo\\[0.5\baselineskip] % Publisher logo
2017 % Year published

\endgroup}



\begin{document}
%\maketitle
\titleBC

% \begin{center}
%   \fbox{\parbox{0.9\textwidth}{\centering
%         Answer the exercises in the spaces provided on the
%         exercise sheets. If you run out of room for an answer,
%         continue on the back of the page.}}
% \end{center}

% \vspace{0.5cm}
% \makebox[0.9\textwidth]{Name:\enspace\hrulefill}


%\begin{exercises}
	
\chapter{Mapping UML Designs to Code \\ Structural Aspects}


During Software Construction, the mapping between design models and source code is essential. 
Use the concepts introduced during the lectures to propose the implementation of the design models below.

\begin{exercise}\textbf{Type Correspondence}

\begin{inparaenum}[(A)]
	\item First, complete Table~\ref{tab:monovalued} to propose a correspondence between UML and Java types for mono-valued attributes.
	\item Second, complete Table~\ref{tab:multivalued}, to propose a similar correspondence table, but for multi-valued attributes. Use the classes and interfaces provided by the Java Collection Framework (JCF).
\end{inparaenum}

\begin{table}
	\begin{center}
		\begin{tabular}{p{5cm}p{5cm}}
			\toprule
			\textbf{UML} & \textbf{Java}\\
			\midrule
String   & \\
Integer  & \\
Real 	  & \\
Boolean  & \\
UnlimitedNatural & \\
			\bottomrule
		\end{tabular}
	\end{center}
	\caption{Types for mono-valued attributes}
	\label{tab:monovalued}
\end{table}



\begin{table}
	\begin{center}
		\begin{tabular}{p{7cm}p{5cm}}
			\toprule
			\textbf{UML} & \textbf{Java}\\
			\midrule
			             \emph{AnyType} [*] \{unique, ordered\}  & \\
			              \emph{AnyType} [*] \{unique, unordered\}   & \\
			              \emph{AnyType} [*] \{nonunique, ordered\}  	  & \\
						  \emph{AnyType} [*] \{nonunique, unordered\}   & \\
			\bottomrule
		\end{tabular}
			\end{center}
			\caption{Types for multi-valued attributes}
			\label{tab:multivalued}
		\end{table}
\end{exercise}




\begin{exercise}
	\textbf{Attribute Implementation (Getter/Setter approach)} 
	
Use Java to implement the class \code{Event} and its attributes. 
Use the Getter/Setter implementation strategy introduced during the lectures.  


\begin{figure}[htbp]
	\centering
		\includegraphics[width=.4\linewidth]{CD-Event.pdf}
	\caption{The Event Class}
	\label{fig:event}
\end{figure}	

\begin{inparaenum}[(A)]
	\item First, declare the Java class and its fields.
	\item Now, declare the class constructor. Remember that some attributes are mandatory and others are not.
	\item Finally, declare the fields' accessors and modifiers (getters and setters). Remember to respect the attribute visibility.
	\item Now, decide how to deal with errors, for instance a start date that is later than the end date. 
\end{inparaenum}
\end{exercise}

\begin{solution}
			There are several ways to deal with errors in Java. In my solution I used assertions, which is a simple solution, but not adapted to public methods. Using runtime exceptions, such as \code{IllegalArgumentException} for the dates and \code{IndexOutOfBoundsException} for multivalued attributes is a more ``java-oriented'' approach. Another interesting approach is the use of AssertJ, which helps developers to specify simple and elegant assertions. Guava, Apache commons, and Lombok are also interesting alternatives to validate parameters.
			
			Alternatives: create our own exceptions, use normal exceptions (not Runtime). Use integers (c-oriented approach.)
%\lstset{language=Java}

\lstinputlisting[language=Java]{./src/main/java/fr/unantes/event/Event.java}

\end{solution}
	
	
\begin{exercise}
	\textbf{Attribute Implementation (Wrapper approach)}
	
Now, use the Wrapper implementation strategy introduced during the lectures to implement the Event class.
	\begin{inparaenum}
		\item For dealing with the multi-valued attribute \code{alarm}, define a wrapper class that is able to check its multiplicity.
		\item Do the same for the mono-valued attributes. Use Java generic types.
	\end{inparaenum}
\end{exercise}
\begin{solution}
	\lstinputlisting[language=Java]{./src/main/java/fr/unantes/event/Event.java}
	\lstinputlisting[language=Java]{./src/main/java/fr/unantes/event/MonovaluedAttribute.java}
	\lstinputlisting[language=Java]{./src/main/java/fr/unantes/event/ReadOnlyMonovaluedAttribute.java}
	\lstinputlisting[language=Java]{./src/main/java/fr/unantes/event/MultivaluedAttribute.java}
\end{solution}


\begin{exercise}
	\textbf{Unidirectional Association Implementation}

Figures~\ref{fig:unidirectional} and~\ref{fig:unique} present two different versions of a unidirectional association between the classes \code{Window} and \code{Field}.
Use Java to implement the classes \code{Window} and \code{Field}, as well as the association between both.

\begin{figure}[htbp]
	\centering
	\includegraphics[scale=.8]{CD-WindowFieldUni.pdf}
	\caption{Unidirectional Association}
	\label{fig:unidirectional}
\end{figure}

\begin{figure}[htbp]
	\centering
		\includegraphics[scale=.8]{CD-WindowFieldUnique.pdf}
	\caption{Unidirectional Unique Association}
	\label{fig:unique}
\end{figure}



\begin{inparaenum}[(A)]
	\item First, for dealing with unidirectional multivalued association roles, define a generic class named \code{Reference} that could be reused by other implementations.
	\item Make your class flexible enough to handle unique et non unique association roles.
	\item Then, declare the Java classes and its fields.
\end{inparaenum}

\end{exercise}

\begin{solution}
		\lstset{language=Java}
		\lstinputlisting{./src/main/java/fr/unantes/unidirectional/Window.java}
		\lstinputlisting{./src/main/java/fr/unantes/unidirectional/Field.java}	
\end{solution}



\begin{exercise}
	\textbf{Bidirectional Association Implementation}

The UML class model presented in Figures~\ref{fig:readOnly} and~\ref{fig:bidirectional}
are similar to the previous ones, except that the association between both classes are bidirectional.

Use Java to implement the classes \code{Window} and \code{Field}, as well as the association between both.
Remember that you must handle the handshake problem.

\begin{figure}[htbp]
	\centering
		\includegraphics[scale=.8]{CD-WindowFieldReadOnly.pdf}
	\caption{Bidirectional Read-Only Association}
	\label{fig:readOnly}
\end{figure}

\begin{figure}[htbp]
	\centering
		\includegraphics[scale=.8]{CD-WindowField.pdf}
	\caption{Bidirectional Association}
	\label{fig:bidirectional}
\end{figure}	


\begin{inparaenum}[(A)]
	\item First, define a class named \code{ReferenceToWindow} that will handle the mono-valued association role \code{container}, from \code{Field} to \code{Window}. This class must implement the methods \code{get()}, \code{set()}, and \code{unset}.
	\item Then, define a class named \code{ReferenceToField} that will handle the multivalued association role \code{widgets}, from \code{Window} to \code{Field}. This class must implement the methods \code{add()}, \code{remove()}, and \code{contains()}.
	\item Now, add the code needed to handle the handshake. 
	\item Finally, declare the Java classes \code{Window} and \code {Field}, and their fields.
\end{inparaenum}
	
\end{exercise}

\begin{solution}
		\lstset{language=Java}
		\lstinputlisting{./src/main/java/fr/unantes/bidirectional/Window.java}
		\lstinputlisting{./src/main/java/fr/unantes/bidirectional/Field.java}	
\end{solution}

\begin{exercise}
	\textbf{Wrapping Things Up}
	
	Use Java to implement the diagram depicted by Figure~\ref{fig:team}.

	\begin{figure}[htbp]
		\centering
			\includegraphics[width=.8\linewidth]{Team}
		\caption{UML Class Diagram\--- Team}
		\label{fig:team}
	\end{figure}
	
\end{exercise}

\chapter{Mapping UML Designs to Code \\ Behavioral Aspects}

\section{Implementing Operations from pre- and post-conditions specifications}

	Consider the class \code{Interval}, illustrated by Figure~\ref{fig:interval}.
	This class has one invariant (Listing~\ref{lst:interval:inv}) and two operations,
	\code{overlapsWith()} (Listing~\ref{lst:overlaps}) and \code{includes()} (Listing~\ref{lst:includes}).

\begin{figure}[htbp]
	\centering
		\includegraphics[width=.4\linewidth]{Interval.png}
	\caption{The class «Interval»}
	\label{fig:interval}
\end{figure}


\lstset{caption={Interval invariant},label=lst:interval:inv,float=htbp}
\begin{ocl}
context Interval(T)
inv: self.end > self.begin
\end{ocl}

\lstset{caption={Operation overlapsWith()},label=lst:overlaps,float=htbp}
\begin{ocl}
context Interval(T)::overlapsWith(other : Interval(T)):Boolean
-- This interval overlaps with another if it includes the others begin or end values
-- or if the other contains this interval begin value.
post: result = self.includes(other.begin) or self.includes(other.end) or
			other.includes(self.begin)
\end{ocl}


\lstset{caption={Operation includes()},label=lst:includes,float=htbp}
\begin{ocl}
context Interval(T)::includes(value : T):Boolean
-- This interval includes a value if it is between begin and end.
post: result = value >= begin and value <= end
\end{ocl}


\begin{exercise}
	First, use JUnit to write some unit tests that verify that the class invariant is correctly respected as well as the correction of the tool operations.
\end{exercise}

\begin{solution}
	\lstset{language=Java}
	\lstinputlisting[language=Java]{./src/test/java/fr/unantes/agenda/IntervalTest.java}
\end{solution}

\begin{exercise}
	Second, implement the class «Interval» and its two attributes, without the two operations.
	Propose an approach to ensure that the class invariant will never be violated.
\end{exercise}

\begin{solution}
	Sorry no correction for now, but I guess the invariant can be checked in the constructor and both attributes could be final.
\end{solution}

\begin{exercise}
	 Finally, implement the operations \code{includes()} and \code{overlapsWith()}, respecting the post-conditions specified above.
\end{exercise}

\begin{solution}
	\lstset{language=Java, caption={Class Interval},}
			\lstinputlisting[language=Java]{./src/main/java/fr/unantes/agenda/Interval.java}
\end{solution}

\newpage
\section{Implementing Operations from Activity diagrams}

Consider the classes \code{Event}, \code{SingleEvent} and \code{RecurrentEvent} illustrated by Figure~\ref{fig:recurrent}.
The class \code{Event} has only one operations, \code{conflictsWith()}, whose algorithm is given by Figure~\ref{fig:conflicts}.
The goal of this operation is to check whether two events happen at the same time. 
This is rather simple for single events, but complex for recurrent events.


\begin{figure}[htbp]
	\centering
		\includegraphics[width=.8\linewidth]{cd-recurrent-event.png}
	\caption{Classes «Event», «SingleEvent», and «RecurrentEvent»}
	\label{fig:recurrent}
\end{figure}


\begin{figure}[htbp]
	\centering
		\includegraphics[width=\linewidth]{ad-conflicts.png}
	\caption{Activity Diagram for «conflictsWith» operation}
	\label{fig:conflicts}
\end{figure}


\begin{exercise}
	First, use a «double dispatch» mechanism to choose the correct implementation of the \code{conflictsWith()} operation.
	
	\begin{inparaenum}[(A)]
		\item Add the methods \code{conflictsWithSingleEvent()} and \code{conflictsWithRecurrentEvent()} to the \code{Event} class.
		\item Implement the methods \code{SingleEvent::conflictsWith()} and \code{RecurrentEvent::conflictsWith()} and make them call the correct implementation methods.
	\end{inparaenum}
\end{exercise}

\begin{solution}
\begin{enumerate}
	\item In the implementation I use the \code{java.time} package. It uses date and time without considering time zones.
	\item The code is still incomplete: the intersection method is still Work TODO.
	\item I'm not a big fan of the double dispatch: it makes the superclass depend on subclasses (bad practice).
	
\end{enumerate}
\end{solution}

\begin{exercise}
Add a method named \code{occurrences()} to class \code{RecurrentEvent}.
This method must generate all occurrences of a recurrent event. 
\end{exercise}

\begin{solution}
\lstset{language=Java, caption={Operation conflictsWith()},}
\lstinputlisting{./src/main/java/fr/unantes/agenda/Event.java}
\lstinputlisting{./src/main/java/fr/unantes/agenda/Frequency.java}
\lstinputlisting{./src/main/java/fr/unantes/agenda/AbstractEvent.java}
\lstinputlisting{./src/main/java/fr/unantes/agenda/SingleEvent.java}
\lstinputlisting{./src/main/java/fr/unantes/agenda/RecurrentEvent.java}
\lstinputlisting{./src/main/java/fr/unantes/agenda/TimeSlot.java}
\end{solution}

\begin{exercise}
Finally, implement the 3 \code{conflictsWith()} methods: one that compares 2 single events, one that compares a single event with
a recurrent event, and one that compares two recurrent events.
\end{exercise}

\chapter{Refactoring to Patterns}

\section{Code Simplification}
\subsection{Introduce Compose Method}
The ``Compose Method\footnote{\url{http://c2.com/ppr/wiki/WikiPagesAboutRefactoring/ComposedMethod.html}}'' pattern is about producing methods that efficiently communicate what they do and how they do what they do. 
According to Kent Beck:

\begin{quotation}
	«Divide your program into methods that perform one identifiable task. Keep all of the operations in a method at the same level of abstraction. 
	This will naturally result in programs with many small methods, each a few lines long.»
\end{quotation}

A Composed Method consists of calls to well-named methods that are all at the same level of detail.
Follow the instructions below to simplify the method \code{add()}~\cite{Kerievsky:2004}:

\lstset{language=Java, caption={Class ArrayList}}
\lstinputlisting{./src/main/java/fr/unantes/refactorings/ArrayList.java}

\begin{exercise}
	Invert the ``readonly'' check to a ``Guard Clause''.
\end{exercise}

\begin{solution}
\begin{lstlisting}[caption=cap,label=lst:lab]
    if (readOnly) {
        return;
    }
\end{lstlisting}

\end{solution}

\begin{exercise}
	Apply the ``Extract Method'' operation to lines 19-20 and create the method \code{void addElement(object)}.
\end{exercise}

\begin{exercise}
	Apply the ``Extract Constant'' operation to replace the magic number ``10 '' and introduce an ``Explaining Variable'' called \code{GROWTH_INCREMENT}.
\end{exercise}

\begin{exercise}
	Apply the ``Inline variable'' operation to \code{newSize} and then apply the ``Extract Method'' operation again to the code that checks whether the element array is at its capacity and needs to grow, and create the method \code{atCapacity()}.
\end{exercise}

\begin{exercise}
	Finally, apply the ``Extract Method'' operation to the code that grows the size of the \code{elements} array, creating the \code{grow()} method.
\end{exercise}

\begin{solution}
	\lstset{language=Java, caption={Class ArrayList after Refactoring}}
	\lstinputlisting{./src/main/java/fr/unantes/refactorings/ArrayListRefactored.java}
\end{solution}

\newpage

\subsection{Replace Conditional Logic with Strategy}

\url{http://www.informit.com/articles/article.aspx?p=1398607&seqNum=2}


\lstset{caption=The Loan Class,label=lst:loan,float=htbp}
\lstinputlisting{./src/main/java/fr/unantes/refactorings/Loan.java}
	

\begin{exercise}
	First, create a class called \code{CapitalStrategy} and add a method called \code{capital()}, with an empty body. 
\end{exercise}
\begin{solution}
\begin{lstlisting}[caption=cap,label=lst:lab,float=htbp]
public class CapitalStrategy {
    public double capital(Loan loan) {
        return 0.0;
    }
}
\end{lstlisting}
\end{solution}

\begin{exercise}
Now, apply the ``Move Method'' refactoring operation to move the \code{capital()} method to class \code{CapitalStrategy}.  This involves:

\begin{enumerate}
	\item Changing the visibility of some private methods: \code{getUnusedPercentage()}, \code{outstandingRiskAmount()}, \code{riskFactor()}, \code{unusedRiskAmount()}, and \code{unusedRiskFactor()}.
	\item Encapsulating fields \code{commitment}, \code{expiry}, \code{maturity}, and \code{payments}.
	\item Creating a simple version of method \code{capital()} on class \code{Loan}, which delegates to an instance of \code{CapitalStrategy}.
	\item Moving the method and replacing all references to \code{this} by \code{loan}.
\end{enumerate}

\end{exercise}
\begin{solution}
\begin{lstlisting}[caption=cap,label=lst:lab,float=htbp]
public class LoanRefactored {
    // (...)

    public double capital() {
        return new CapitalStrategy().capital(this);
    }
    // (...)
    protected double riskFactor() {
        return RiskFactor.getFactors().forRating(riskRating);
    }

    protected double unusedRiskFactor() {
        return UnusedRiskFactors.getFactors().forRating(riskRating);
    }

    protected double getUnusedPercentage() {
        return 0.0;
    }

    protected Date getMaturity() {
        return maturity;
    }

    protected Date getExpiry() {
        return expiry;
    }

    protected double getCommitment() {
        return commitment;
    }

    protected List<Payment> getPayments() {
        return payments;
    }
}

public class CapitalStrategy {
    public double capital(LoanRefactored loan) {
        if (loan.getExpiry() == null && loan.getMaturity() != null)
            return loan.getCommitment() * loan.duration() * loan.riskFactor();
        if (loan.getExpiry() != null && loan.getMaturity() == null) {
            if (loan.getUnusedPercentage() != 1.0)
                return loan.getCommitment() *loan.getUnusedPercentage() * loan.duration() * loan.riskFactor();
            else
                return (loan.outstandingRiskAmount() * loan.duration() * loan.riskFactor())
                        + (loan.unusedRiskAmount() * loan.duration() * loan.unusedRiskFactor());
        }
        return 0.0;
    }
}
\end{lstlisting}
\end{solution}

\begin{exercise}
	Since some methods, such as \code{duration()}, \code{weightedAverageDuration()}, \code{yearsTo()},  \code{riskFactor()} and \code{unusedRiskFactor()} are only used by method \code{capital()}, we can move them to class \code{CapitalStrategy} as well.
	The two constants, \code{MILLIS_PER_DAY} and \code{DAYS_PER_YEAR} can also be moved to class  \code{CapitalStrategy}.
\end{exercise}

\bibliographystyle{abbrv}
\bibliography{references}
\end{document}


